
\documentclass[oneside,a4paper]{memoir} % A4 paper size, default 11pt? font size and oneside for equal margins

\usepackage{./sty/style_report}




% Glossary
\newglossaryentry{}{name=,
  description={}}

% Acronym
\setacronymstyle{long-short}
\newacronym{AU}{AU}{Aarhus University}
\newacronym{SDAM}{SDAM}{Social Dynamics in the Ancient Mediterranean}
\newacronym{EDH}{EDH}{Epigraphic Database Heidelberg}
\newacronym{API}{API}{Application Programming Interface}
\newacronym{URI}{URI}{Uniform Resource Identifier}
\newacronym{JSON}{JSON}{Java Script Object Notation}
%\newacronym{DTAP}{DTAP}{development $\to$ test $\to$ acceptance $\to$ production}

\makeglossaries
%\loadglsentries{acronyms}

\begin{document}


\chapter{Short report: Epigraphic Database Heidelberg using \textsf{R}}
\chapterauthor{Antonio Rivero Ostoic \\ \small 28-11-2019}
%\vfil
This post is about accessing the \enquote{Epigraphic Database Heidelberg} (EDH), which is one of the longest running database projects in digital Latin epigraphy. 
The \gls{EDH} database started as early as year 1986, and in 1997 the Epigraphic Database Heidelberg website was launched at \url{https:/edh-www.adw.uni-heidelberg.de} where inscriptions, images, bibliographic and geographic records can be searched and browsed online.

Despite the possibility of accessing the \gls{EDH} database through a Web browser, it is many times convenient to get the Open Data Repository by the \gls{EDH} through its public Application Programming Interface (API). 

\noindent
For inscriptions, the generic search pattern Uniform Resource Identifier (URI) is:
\begin{lstlisting}
https://edh-www.adw.uni-heidelberg.de/data/api/inscriptions/search?par_1=value&par_2=value&par_n=value
\end{lstlisting}
with parameters \code{par} $1,2,...n$. 

\medbreak
\noindent
The response from a query is in a Java Script Object Notation (JSON) format such as:
\begin{lstlisting}
{
   "total" : 61,
   "limit" : "20",
   "items" : [ ... ]
}
\end{lstlisting}



\section{Accessing the EDH database using R}
Accessing the \gls{EDH} database \gls{API} using \textsf{R} is possible with a convenient function that produces the generic search pattern \gls{URI}. 
Hence, the function \code{get.edh()} allows having access to the data with the available parameters that are recorded as arguments. Then the returned \gls{JSON} file is converted into a list data object with function \code{fromJSON()} from the \textsf{rjson} package. 

Basically, the function \code{get.edh()} allows getting data with the \code{search} parameter either from \code{"inscriptions"} (the default option) or else from \code{"geography"}. 
The other two options from the \gls{EDH} database \gls{API}, which are \code{"photos"} and \code{"bibliography"} may be implemented in the future. 

\bigbreak
\noindent
The following parameter description is from \url{https://edh-www.adw.uni-heidelberg.de/data/api}:

\subsection*{Search parameters for inscriptions and geography}
\begin{description}\small
  \item[\code{province}]   get list of valid values at \url{https://edh-www.adw.uni-heidelberg.de/data/api/terms/province}, case insensitive
  \item[\code{country}]   get list of valid values at \url{https://edh-www.adw.uni-heidelberg.de/data/api/terms/country}, case insensitive
  \item[\code{findspot\_modern}]   add leading and/or trailing truncation by asterisk *, e.g. findspot\_modern=k�ln*, case insensitive
  \item[\code{findspot\_ancient}]   add leading and/or trailing truncation by asterisk *, e.g. findspot\_ancient=aquae*, case insensitive
  \item[\code{bbox}]   bounding box in the format bbox=minLong, minLat, maxLong, maxLat, example: \url{https://edh-www.adw.uni-heidelberg.de/data/api/inscriptions/search?bbox=11,47,12,48}
\end{description}

\noindent
Just make sure to quote the arguments in \code{get.edh()} for the different parameters that are not integers. 
This means for example that the query for the last parameter with the two search options is written as 
\begin{lstlisting}
R> get.edh(search="inscriptions", bbox="11,47,12,48")
R> get.edh(search="geography", bbox="11,47,12,48")
\end{lstlisting}


\subsection*{Search parameters for inscriptions}
\begin{description}\small
  \item[\code{hd\_nr}]  HD-No of inscription
  \item[\code{year\_not\_before}]  integer, BC years are negative integers
  \item[\code{year\_not\_after}]   integer, BC years are negative integers
  \item[\code{tm\_nr}]  integer value (?)
  \item[\code{transcription}]  automatic leading and trailing truncation, brackets are ignored
  \item[\code{type}]  of inscription, get list of values at \url{https://edh-www.adw.uni-heidelberg.de/data/api/terms/type}, case insensitive
\end{description}


\subsection*{Search parameters for geography}
\begin{description}\small
  \item[\code{findspot}]  level of village, street etc.; add leading and/or trailing truncation by asterisk *, e.g. findspot\_modern=koln*, case insensitive
  \item[\code{pleiades\_id}]  Pleiades identifier of a place; integer value
  \item[\code{geonames\_id}]  Geonames identifier of a place; integer value
\end{description}

\noindent
Since with the \code{"inscriptions"} option the \code{id} \enquote{component} of the output list is not with a numeric format, then function \code{get.edh()} adds an \code{ID} at the beginning of the list with the identifier with a numerical format. 

\bigbreak
\noindent
Hence, the query 
\begin{lstlisting}
R> get.edh(findspot_modern="madrid")
\end{lstlisting}
returns this truncated output:
\begin{verbatim}
$ID
[1] "041220"

$commentary
[1] " Verschollen. Mogliche Datierung: 99-100."

$country
[1] "Spain"

$diplomatic_text
[1] "[ ] / [ ] / [ ] / GER PO[ ]TIF / [ ] / [ ] / [ ] / ["

...
\end{verbatim}

\noindent
Having a numerical identifier is useful for plotting the results for example. However, it is possible to prevent this addition by disabling argument \code{addID} with \code{FALSE}.  
\begin{lstlisting}
R> get.edh(findspot_modern="madrid", addID=FALSE)
\end{lstlisting}


\bigbreak
\noindent
Finally, it is worth to mention that further extensions that the \gls{EDH} database \gls{API} may add in the future can be handled with similar arguments in function \code{get.edh()}. 

%\subsubsection{Human testing}

\printglossary

\end{document}

% padflatex reportNOV19.tex
% makeglossaries reportNOV19
% padflatex reportNOV19.tex
